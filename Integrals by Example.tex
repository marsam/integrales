
%%%%%%%%%%%%%%%%%%%%%%%%%%%%%%%%%%%%%%%%%%%%%%%%%%%%%%%%%%%%%%%%%%%%%%%%%
%%%%%%%%%%%%%%%%%%%%%%%%%%%%%%%%%%%%%%%%%%%%%%%%%%%%%%%%%%%%%%%%%%%%%%%%%
% V.0.2
% -Mario Rodas 
%%%%%%%%%%%%%%%%%%%%%%%%%%%%%%%%%%%%%%%%%%%%%%%%%%%%%%%%%%%%%%%%%%%%%%%%%
%%%%%%%%%%%%%%%%%%%%%%%%%%%%%%%%%%%%%%%%%%%%%%%%%%%%%%%%%%%%%%%%%%%%%%%%%

\documentclass[12pt,a4paper]{scrartcl}
\usepackage{amsmath}
\usepackage{amssymb,amsfonts,amsthm}
\linespread{1.2}
\usepackage[latin1]{inputenc}
\usepackage[spanish]{babel}
\usepackage[pdftex]{graphicx} %graficos
\usepackage{url} %url
\usepackage[pdftex,breaklinks=true]{hyperref} %links

%%%%%%%%%%%%%%%%%%%%%%%%%%%%%%%%%%%%%%%%%%%%%%%%%%%%%%%%%%%%%%%%%%%%%%%%%
%%%%%%%%%%%%%%%%%%%%%%%%%%%%%%%%%%%%%%%%%%%%%%%%%%%%%%%%%%%%%%%%%%%%%%%%%

%Teoremas, corolarios y lemas. Remarks y definiciones
\newtheorem{thm}{Teorema}[section]
\newtheorem{cor}[thm]{Corolario}
\newtheorem{lem}[thm]{Lema}


\theoremstyle{remark}
\newtheorem{rem}[thm]{Remark}

\theoremstyle{definition}
\newtheorem{defin}[thm]{Definici�n}

\theoremstyle{remark}
\newtheorem{ejm}[thm]{Ejemplo}


%Nuevo parrafo con interlineado. Cabeceras y titulos con estilo article
\newcommand\mypar{\par\vspace{\baselineskip}}
\renewcommand{\sectfont}{\rmfamily \bfseries} %cabeceras y titulos con estilo roman, negrita.

%Nuevos comandos
%\newcommand{\inv}[1]{\frac{1}{#1}}
%Vamos, no deseo ensuciarme la manos...mas comandos

\newcommand{\arctanh}{\textnormal{arctanh}}
\newcommand{\arcsenh}{\textnormal{arcsenh}}
\newcommand{\arccosh}{\textnormal{arccosh}}
\newcommand{\arccoth}{\textnormal{arccoth}}
\newcommand{\arcsech}{\textnormal{arcsech}}
\newcommand{\arccsch}{\textnormal{arccsch}}
\newcommand{\sech}   {\textnormal{sech}}
\newcommand{\csch}   {\textnormal{csch}}
\newcommand{\ctanh}  {\textnormal{ctanh}}
\newcommand{\arccot} {\textnormal{arccot}}
\newcommand{\arcsec} {\textnormal{arcsec}}
\newcommand{\arccsc} {\textnormal{arccsc}}
\newcommand{\erf}    {\textnormal{erf}}

%%%%%%%%%%%%%%%%%%%%%%%%%%%%%%%%%%%%%%%%%%%%%%%%%%%%%%%%%%%%%%%%%%%%%%%%%
%%%%%%%%%%%%%%%%%%%%%%%%%%%%%%%%%%%%%%%%%%%%%%%%%%%%%%%%%%%%%%%%%%%%%%%%%

%Title
\title{\normalfont{Integrals by Example}} %Titulo del documento con estilo article
\author{\copyright Mario Rodas Palomino}
\date{2 de setiembre del 2008}

%%%%%%%%%%%%%%%%%%%%%%%%%%%%%%%%%%%%%%%%%%%%%%%%%%%%%%%%%%%%%%%%%%%%%%%%%
%%%%%%%%%%%%%%%%%%%%%%%%%%%%%%%%%%%%%%%%%%%%%%%%%%%%%%%%%%%%%%%%%%%%%%%%%

%%%%%%%%%%%%%%%%%%%%%%%%%%%%%%%%%%%%%%%%%%%%%%%%%%%%%%%%%%%%%%%%%%%%%%%%%
%%%%%%%%%%%%%%%%%%%%%%%%%%%%%%%%%%%%%%%%%%%%%%%%%%%%%%%%%%%%%%%%%%%%%%%%%

\begin{document}
\maketitle
%\newpage
%\section*{Aclaraci\'on}
%Este documento esta en \textsl{beta} cualquier error, \textbf{guardarlo en secreto}.
\begin{center}
 \begin{minipage}{0.4\textwidth}
		\includegraphics{by-nc-sa}
 \end{minipage}%
 \begin{minipage}{0.4\textwidth}
  \begin{flushright}
    \small
       	Echo con \LaTeXe + Vim como Dios manda.
  \end{flushright}
 \end{minipage}
\end{center}
\section{Introducci\'on}
Hist\'oricamente, el concepto de integral de una funci\'on surgi\'o en relaci\'on con dos problemas aparentemente distintos, pero en realidad \'intimamente relacionados\footnote{Ve\'ase: Teorema fundamental del C\'alculo}. Por un lado, la \emph{integral definida} de una funci\'on $f:\mathbb{R} \rightarrow \mathbb{R}$ en un intervalo $\left[a,b\right]$ (en el cual $f$ est\'a definida y no es negativa) se define como el \'area comprendida entre la gr\'afica de la funci\'on y el eje de las abscisas en el intervalo $\left[a,b\right]$. Por otro lado la \emph{integral indefinida} \'o \emph{primitiva} de $f$ es cualquier funci\'on derivable cuya derivada es $f$. Algo que debo de aclarar es que si $f$ es una funci\'on integrable en $\left[a,b\right]$ , $f$ no tiene por qu\'e tener una primitiva.
El objeto de estos apuntes no es definir anal\'iticamente la \emph{integral definida} sino el elaborar m\'etodos para hallar la integral indefinida en general trabajaremos con la integral indefinida
$$\int f\left(x\right) dx$$
Cuya soluci\'on tiene la forma $g\left(x\right) + C$ , donde $g$ es una funci\'on que satisface\footnote{Utilizaremos esta condici\'on para verificar que $g$ es una soluci\'on de la integral} :
$$g'\left(x\right) = f\left(x\right)$$
El proceso de hallar la soluci\'on de una integral es lo que se denomina \emph{integrar una funci\'on}, o simplemente \emph{integraci\'on}. En la expresi\'on $\int f\left(x\right) dx = g\left(x\right) + C$, la funci\'on $f\left(x\right)$ se llama \emph{integrando} y la funci\'on $g\left(x\right)$ se llama \emph{primitiva} o \emph{antiderivada} de $f\left(x\right)$ y $C$ es la \emph{constante de integraci\'on} la cual olvidaremos escribir en general (muchas veces por motivos de espacio). Sin embargo, se debe tener siempre presente que son infinitas las soluciones de una integral indefinida y cualquier par de difieren en una constante.
�Es importante conocer la primitiva de una funci\'on? S\'i, su importancia se deriva de la regla de Barrow%El Teorema fundamental del C\'alculo relaciona ambos conceptos, estableciendo(bajo ciertas condiciones t\'ecnicas) que la funci\'on cuyo valor en $x$

%%%%%%%%%%%%%%%%%%%%%%%%%%%
%%%%%%%%%%%%%%%%%%%%%%%%%%%

\section{Un poco de teor\'ia}
\begin{defin}[Definici\'on de primitiva]Llamaremos primitiva (o integral indefinida) de una funci\'on $f: \mathbb{R} \rightarrow \mathbb{R}$ definida en un intervalo $J$ a cualquier funci\'on $g: \mathbb{R} \rightarrow \mathbb{R}$ tal que $g'=f$ en $J$
\end{defin}
Donde estamos utilizando la convenio de escribir $g'\left(x\right)$ por $g'\left(x + 0\right)$ \'o $g'\left(x-0\right)$) si $x$ es uno de los extremos de $J$.
\begin{thm}
Si $g_1$ y $g_2$ son primitivas de la funci\'on $f$ en un intervalo $J$ . Entonces $g_1 -g_2 = constante$
\end{thm}
En efecto si $g_2$ es otra primitiva de $f$ entonces $\left(g_1 - g_2\right)' = f -f = 0$ en $J$.

%%%%%%%%%%%%%%%%%%%%%%%%%
%%%%%%%%%%%%%%%%%%%%%%%%% 

\subsection{Propiedades}
Estas son algunas propiedades de las primitivas (donde $F$ es la primitiva de $f$) :
\begin{align}
	&\int \lambda f\left(x\right) \pm \mu g\left(x\right) dx = \lambda \int f\left(x\right)dx \pm \mu \int g\left(x\right) dx && \text{(linealidad)} \nonumber \\
	&\int f\left(x\right) g\left(x\right) dx = F\left(x\right)g\left(x\right) - \int F\left(x\right) g'\left(x\right) dx && \text{(integraci\'on por partes)} \nonumber	\\
	&\int f\left(g\left(x\right)\right) g'\left(x\right) dx = \int F\left(g\left(x\right)\right) && \text{(sustituci\'on)} \nonumber \\
	&\int f\left(ax + b\right) dx = \frac{1}{a} F\left(ax + b\right) \nonumber 
\end{align}

Acerca de la linealidad de la integrales\footnote{La cual se puede deducir facilmente de la igualdad de derivadas: $\left(\lambda f\left(x\right) + \mu g\left(x\right)\right)' =\lambda f'\left(x\right) + \mu g'\left(x\right)$}, si $f$ y $g$ admiten una primitiva en un intervalo $J$, entonces otro tanto ocurre con $\lambda f\left(x\right) +\mu g\left(x\right)$ para todo $\lambda , \mu \in \mathbb{R}$ . N\'otese que la existencia de la primitiva de (por ejemplo) de $f + g$ no implica la existencia de la primitiva de $f$ y $g$ por separado.

\subsection{Integrales b\'asicas}\label{tabla}
\begin{eqnarray}
	&& \int x^n dx = \frac{x^{n+1}}{n+1} + C \qquad, \forall n\in \mathbb{R} / n\neq -1 \\
	&& \int \frac{1}{x}dx = \ln|x| + C\\
	&& \int e^{x}dx=e^{x} + C \\
	&& \int \sen x dx=-\cos x + C \\
	&& \int \cos x dx=\sen x + C \\
	&& \int \sec^2 x dx=\tan x + C \\
	&& \int \csc^2 x dx=-\cot x + C \\
	&& \int \left(\sec x \right)\left(\tan x \right) dx=\sec x + C \\
	&& \int \left(\csc x \right)\left(\cot x \right) dx=-\csc x+ C \\
	&& \int \frac{1}{\sqrt{1-x^2}} dx=\arcsen x + C=-\arccos x + C \qquad , |x|<1 \\
	&& \int \frac{1}{1+x^2}dx =\arctan x + C=-\arccot x + C \\
	&& \int \frac{1}{x\sqrt{x^2 +1}}dx =\arcsec |x| + C=-\arccsc |x| + C \qquad , |x|>1 \\
	&& \int \senh x dx =\cosh x + C \\
	&& \int \cosh x dx =\senh x + C \\
	&& \int \sech^2 x dx =\tanh x + C \\
	&& \int \csch^2 x dx =-\coth x + C \\
	&& \int \frac{1}{\sqrt{1 + x^2}} dx =\arcsenh x + C = \ln\left(x + \sqrt{1 + x^2}\right) + C \\
	&& \int \frac{1}{\pm\sqrt{x^2 - 1}} dx =\arccosh x + C = \ln\left(x \pm \sqrt{x^2 - 1}\right) + C \qquad, |x|>1
\end{eqnarray}	
\begin{equation}
\int \frac{1}{1-x^2} dx = {\left\{ \begin{array}{ll}
						\arctanh x = \frac{1}{2}\ln{\frac{1+x}{1-x}}\quad , |x|<1\\
						\arccoth x = \frac{1}{2}\ln{\frac{1+x}{x-1}}\quad , |x|>1
					\end{array} \right.}
\end{equation}
%\end{eqnarray}

%%%%%%%%%%%%%%%%%%%%%%%%%%%%%%%%%%%%%%%%%%%%%%%%%%%%%%%%%%%%%%%%%%%%%%%%%
%%%%%%%%%%%%%%%%%%%%%%%%%%%%%%%%%%%%%%%%%%%%%%%%%%%%%%%%%%%%%%%%%%%%%%%%%
%
%         M�todos de Integraci�n
%
%%%%%%%%%%%%%%%%%%%%%%%%%%%%%%%%%%%%%%%%%%%%%%%%%%%%%%%%%%%%%%%%%%%%%%%%%
%%%%%%%%%%%%%%%%%%%%%%%%%%%%%%%%%%%%%%%%%%%%%%%%%%%%%%%%%%%%%%%%%%%%%%%%%

\section{M\'etodos de integraci\'on}

\subsection{Sustituci\'on o cambio de variable}
\'Este m\'etodo es la forma integral de la regla de la cadena\footnote{Si $g : \mathbb{R}\rightarrow\mathbb{R}$ es derivable en $a$ y $f :\mathbb{R}\rightarrow\mathbb{R}$ es derivable en $g\left(x\right)$, entonces $f\circ g$ es derivable en $a$ y se verifica $$\left(f \circ g\right)'\left(a\right)=f'\left(g\left(a\right)\right)g'\left(a\right)$$}:

\begin{thm}\label{thm:ivar} 
Sea $g :\mathbb{R}\rightarrow\mathbb{R}$ una funci\'on con derivada continua en $\left[a,b\right]$, y sea $f : \mathbb{R}\rightarrow\mathbb{R}$ continua en $g\left(\left[a,b\right]\right)$.Si $F$ es una primitiva de $f$ entonces se cumple:
\begin{equation}
\int f\left(g\left(x\right)\right)g'\left(x\right)dx = F\left(g\left(x\right)\right)
\end{equation}
\end{thm}

Este m\'etodo de hallar primitivas efectuando un cambio de variable adecuado es muy poderoso y flexible, pero tiene el incoveniente de depender esencialmente de la habilidad en escoger el cambio de varible indicado en cada caso, lo cual escapa a la sistematizaci\'on.
%
\begin{ejm}\label{sec:Ejm1}
Calcular $\int \cos 2x dx$. Este es un ejemplo sencillo, pero voy a resolverlo para mostrar la simplicidad de este m\'etodo:
$$\int \cos 2x dx = \left\{ \begin{array}{ll} y=2x\\ dy = 2 dx \end{array} \right\} = \int \cos y \: \frac{dy}{2}=\frac{1}{2}\sen y = \frac{1}{2} \sen 2x $$
\end{ejm}
%
\begin{ejm}\label{sec:Ejm2}
Calcular $\int \sqrt{1 - x^2}dx$ . Notamos que la integral no aparece en nuestra tabla de integrales b\'asicas (\ref{tabla}). Es necesario \emph{convertirla} en una de la tabla, vamos a realizar el cambio de varible $x = \sen y$.
	$$\int \sqrt{1 - x^2}dx = \left\{ \begin{array}{ll} x = \sen y \\ dx = \cos y dy \end{array} \right\} = \int \sqrt{1-\sen^2 y} \cos y \: dy = \int \cos^2 y \: dy$$
Esta \'ultima integral se calcula facilmente usando la identidad trigonom\'etrica $\cos^2 \lambda = \frac{\left(1+\cos 2\lambda \right)}{2}$.
$$\int \cos^2 y \: dy = \frac{1}{2}\int \left(1+\cos 2y\right) \: dy = \frac{1}{2}\left(y + \frac{1}{2}\sen 2y\right)=\frac{1}{2}\left(y + \sen \cos y\right)$$
por lo tanto:
\begin{eqnarray}
\left[\int \sqrt{1 - x^2} dx\right]_{x=\sen y} &=& \frac{1}{2}\left(y + \sen y \cos y\right) \nonumber \\
 \Longleftrightarrow \int \sqrt{1 - x^2}dx &=& \frac{1}{2}\left(\arcsen x + x\sqrt{1-x^2}\right) \nonumber 
\end{eqnarray}
Ya que $x = \sen y$ si y solo si $y = \arcsen x$.
\end{ejm}
%
\begin{ejm}\label{sec:Ejm3}
Calcular $\int \sec x \: dx$ . Esta integral es un poco mas complicada que el anterior, y como no aparece en nuestra tabla (\ref{tabla}) vamos a \emph{convertirla} en una de la tabla:
$$\int \sec x \: dx = \int \sec x \underbrace{\left( \frac{\sec x + \tan x}{\sec x + \tan x}\right)}_1 dx = \int\frac{\sec x\tan x +\tan^2 x}{\sec x + \tan x} dx $$
Hacemos el cambio de variable $y = \sec x + \tan x$, luego:
$$\int\frac{\sec x\tan x +\tan^2 x}{\sec x + \tan x} dx = \left\{ \begin{array}{ll} y = \sec x + \tan x \\ dy = \left(\sec x\tan x +\tan^2 x\right)dx \end{array} \right\} = \int \frac{1}{y} \: dy = \ln |y|$$
$$\int \sec x \: dx = \ln{| \sec x + \tan x |}$$
\end{ejm}
%%%%%%%%%%%%%%%%%%%%%%%%%%%%%%%%%%%%%%%%%%%%%%%%%%%%%%%%%%%%%%%%%%%%%%%%%
%%%%%%%%%%%%%%%%%%%%%%%%%%%%%%%%%%%%%%%%%%%%%%%%%%%%%%%%%%%%%%%%%%%%%%%%%

\subsection{Integraci\'on por partes}

\begin{thm}[F\'ormula de Integraci\'on por partes]\label{thm:ipartes} Si $f'\left(x\right)$ y $g'\left(x\right)$ son continuas en un intervalo$\left[a,b\right]$ entonces:
\begin{equation}
\int f\left(x\right)g'\left(x\right) dx = f\left(x\right) g\left(x\right) - \int f'\left(x\right) g\left(x\right) dx
\end{equation}

\end{thm}

La f\'ormula anterior es la versi\'on en terminos de primitivas de la regla de Leibniz.
%
\begin{ejm}
Calcular $\int \arctan x dx$. Para resolver esta integral utilizaremos la integraci\'on por partes:
$$\int \arctan x \: dx = \int \left(x\right)'\arctan x \:dx = x \arctan x -\int \frac{x}{1 + x^2} dx = x \arctan x - \frac{1}{2}\ln{\left(1+x^2\right)}$$
\end{ejm}
%
\begin{ejm}
Calcular $\int x^n \ln x dx$. Para resolver esta integral utilizaremos la integraci\'on por partes:
$$\int x^n \ln x dx = \int \left(\frac{x^{n+1}}{n+1}\right)'\ln x dx =  \frac{x^{n+1}}{n+1}\ln x - \int \frac{x^{n+1}}{n+1} dx = \frac{x^n}{n+1}\left(\ln x -\frac{1}{n+1}\right)$$
\end{ejm}
%
\begin{ejm}
Calcular $I = \int e^{ax} \cos bx \: dx$. Aparentemente mas dif\'icil que el problema anterior:
$$I= \int e^{ax} \cos bx dx = \int e^{ax} \left(\frac{\sen bx}{b}\right)' dx = \frac{e^{ax} \sen bx}{b} - \frac{a}{b} \int e^{ax} \sen bx \: dx$$
La integral $\int e^{ax} \sen{bx} \: dx$ es de la misma forma que la original asi que volvemos a aplicar la integraci\'on por partes:
$$ \int e^{ax} \sen bx \: dx = \int e^{ax} \left(-\frac{\cos bx}{b}\right)' dx = - \frac{e^{ax} \cos bx}{b} + \frac{a}{b} \int e^{ax} \cos bx \: dx$$
Juntando las dos f\'ormulas anteriores
$$I = \frac{e^{ax} \sen bx}{b} + \frac{a}{b^2}e^{ax} \cos{bx} -\frac{a^2}{b^2}I $$
de donde, resolviendo la ecuaci\'on respecto a $I$ obtenemos:
$$I = \int e^{ax} \cos{bx} \: dx = \frac{b \sen{bx} + a \cos{bx}}{a^2 + b^2}e^{ax}.$$
\end{ejm}

Una peque\~na  observaci\'on acerca de la integraci\'on por partes, la primitiva de un polinomio es otro polinomio, que se calcula f\'acilmente utilizando la f\'ormula para la primitiva $x^n$ con $n \in \mathbb{N} \cup 0 $. Si $P$ es un polinomio y $f$ es una funci\'on (por ejemplo) infinitamente diferenciable con una primitiva $g$ conocida, la regles de Leibniz proporciona
$$\int P\left(x\right) f\left(x\right) dx = \int P\left(x\right) g'\left(x\right) dx = P\left(x\right)g\left(x\right) - \int P'\left(x\right) g\left(x\right) dx$$
donde aparece la integral de $g$ por un polinomio de grado inferior en una unidad al de $P$. Si $\int g$ es tambi\'en conocida, se puede aplicar la regla de Leibniz para simplificar la \'ultima integral, y as\'i sucesivamente. En particular si $f\left(x\right)= e^x$ \'o $f\left(x\right)=\sen \left(x\right)$ \'o $f\left(x\right)= \cos \left(x\right)$, todas cuyas primitivas sucesivas son conocidas, es claro que este m\'etodo permite calcular $\int P\left(x\right) f\left(x\right) dx$ para cualquier polinomio $P$. Por ejemplo:
$$\int x^2 e^x dx = x^2 e^x -2\int xe^x dx= x^2 e^x - 2xe^x + 2\int e^x dx = \left(x^2 -2x +2\right)e^x$$

%%%%%%%%%%%%%%%%%%%%%%%%%%%%%%%%%%%%%%%%%%%%%%%%%%%%%%%%%%%%%%%%%%%%%%%%%
%%%%%%%%%%%%%%%%%%%%%%%%%%%%%%%%%%%%%%%%%%%%%%%%%%%%%%%%%%%%%%%%%%%%%%%%%

\subsection{Integraci\'on de funciones racionales}

\begin{defin}
Diremos que una funci\'on racional $f\left(x\right) = \frac{P_n \left(x\right)}{Q_m \left(x\right)}$ es simple si el grado del polinomio $P_n \left(x\right)$ es menor que el del polinomio $Q_m \left(x\right)$ , o sea , si $n < m$.
\end{defin}

Si $n > m$ entonces podemos dividir los polinomios $P_n \left(x\right)$ y $Q_m \left(x\right)$ de tal forma que 
\begin{equation}
\frac{P_n \left(x\right)}{Q_m \left(x\right)} = p_{n-m} \left(x\right) + \frac{R_k \left(x\right)}{Q_m \left(x\right)} , \qquad \textrm{donde} \quad k<m
\end{equation}
%
\begin{thm}[Teorema de descomposici\'on en Fracciones simples] 
Supongamos que  $\frac{P_n \left(x\right)}{Q_m \left(x\right)}$ es una fracci\'on simple , y que el polinomio denominador se pueda factorizar de la siguiente forma \footnote{Esta descomposici\'on siempre es posible debido al \emph{Teorema fundamental del \'Algebra}}
\begin{equation}
Q_n \left(x\right) = c \left(x - x_1 \right)^{n_1} \cdots \left(x - x_p\right)^{n_p} {\left[\left(x - a_1\right)^2 + b_1 ^2\right]}^{m_1} \cdots {\left[\left(x - a_k\right)^2 + b_k ^2\right]}^{m_k}
\end{equation}
Siendo $n_i , m_j \geq 1$ y $b_j > 0$ para todo $i = 1 , \ldots , p$ , $j = 1 , \ldots , k$ . Los numeros $x_i \in \mathbb{R}$ son las raices de $Q_m \left(x\right)$  , de multiplicidad $n_i$ ,y los n\'umeros complejos $a_i \pm i b_j \in \mathbb{C}$ son las raices complejas de dicho polinomio, de multiplicidad $m_j$ .  Entonces la fracci\'on simple $\frac{P_n \left(x\right)}{Q_m \left(x\right)}$  se puede descomponer en la suma de \textbf{fracciones elementales simples} 
\begin{eqnarray}
	\lefteqn{\frac{P_n \left(x\right)}{Q_m \left(x\right)} = \frac{A_{1 1}}{x - x_1} + \ldots + \frac{A_{1 n_1}}{\left(x - x_1\right)^{n_1}} + \ldots }  {} \nonumber\\ 
	&&  \quad + \frac{A_{p 1}}{\left(x - x_p\right)} + \ldots \frac{A_{p n_{p}}}{\left(x - x_p\right)^{n_p}} + \ldots \nonumber\\ 
	&&  \quad + \frac{B_{1 1} x + C_{1 1}}{\left(x - a_1\right)^2 + b_1 ^2} + \ldots + \frac{B_{1 m_1}x + C_{1 m_1}}{\left[\left( x - a_1\right)^2 + b_1 ^2 \right]^{m_1}} + \ldots \nonumber\\
	&&  \quad + \frac{B_{k 1} x + C_{k 1}}{\left(x - a_k\right)^2 + b_k ^2} + \ldots + \frac{B_{k m_k}x + C_{k m_k}}{\left[\left( x - a_k\right)^2 + b_k ^2 \right]^{m_k}} . 
	\end{eqnarray}
donde $A_{ij} , B_{ij} , C_{ij}$ son ciertas contantes reales
\end{thm}
%Este es un importante teorema de \'Algebra y se demuestra en los cursos de Variable Compleja. 
Para determinar dichas constantes sumamos los t\'erminos de la derecha. N\'otese que el denominador cum\'un conincide con la descomposici\'on de $Q_n \left(x\right)$ y el numerador es un polinomio de grado a lo sumo $n$. Luego comparamos el polinomio numerador que se obtiene al sumar las fracciones simples con $P_n \left(x\right)$. Igualando los coeficientes de ambos obtendremos un sistema de $n$ ecuaciones con $n$ inc\'ognitas que podremos resolver para encontrar los coeficientes indeterminados $A_{ij} , B_{ij} , C_{ij}$.
No obstante existen \emph{casos} en los que es posible hallar los coeficientes sin resolver un sistema de ecuaciones, los cuales detallar\'e en la resoluci\'on de problemas. 

Luego de hallar las fracciones parciales, basta por lo tanto saber calcular las siguientes integrales:
\begin{eqnarray}
	&& \int \frac{dx}{\left(x-a\right)^n} = \int \left( x - a \right)^{-n} dx = \frac{\left(x-a\right)^{1-n}}{1-n}, \qquad n\in \mathbb{N}, \: n>1 \nonumber	\\
	&& \int \frac{dx}{x-a} = \ln{|x - a|} \nonumber \\
	&& \int \frac{Cx + D}{\left[\left(x - a\right)^2 + b^2\right]^n} dx, \qquad b \neq 0 \nonumber
\end{eqnarray}
Esta ultima integral se simplifica primero mediante un cambio de variable $x-a = bu$ , que la transforma en una integral mas simple del tipo
$$\int \frac{Au + B}{\left(u^2 + 1 \right)^n} du = A\int \frac{u \: du}{\left(u^2 +1\right)^n} + B\int \frac{du}{\left(u^2 + 1 \right)^n}$$ 
La primera integral se calcula f\'acilmente mediante el cambio $u^2 + 1 = v$ , obteniendose:
\begin{displaymath}
\int \frac{u}{\left(u^2 + 1\right)^n} du = \frac{1}{2} \int v^{-n} dv = \left\{ \begin{array}{ll} \frac{1}{2}\ln{|v|} = \frac{1}{2} \ln{\left(u^2 + 1\right)} , & n=1 \\ 
\frac{v^{1-n}}{2 \left( 1 - n \right)} = \frac{\left(u^2 + 1 \right)^{1-n}}{2 \left( 1 - n \right)} , & n>1  \end{array} \right. 
\end{displaymath}
Por lo tanto, el problema de integrar una funci\'on racional arbitraria se reduce a \'ultima instancia al problema de calcular la integral
$$I_n \left(u\right) = \int \frac{du}{\left(u^2 + 1\right)^n} , \qquad n \in \mathbb{N}$$
Si $n=1$ , ya hemos visto que $I_1 \left(u\right) = \arctan u$. Si $n \in \mathbb{N}$ , integrando por partes obtenemos

\begin{eqnarray}
I_n \left(u\right) &=& u\frac{1}{\left(u^2 +1\right)^n} - \int u \frac{-2nu \:du}{\left( u^2 + 1\right)^{n+1}} \nonumber \\ 
&=& \frac{u}{\left( u^2 + 1\right)^n} + 2n \int \frac{u^2 du}{\left(u^2 +1 \right)^{n+1}}  \nonumber\\  
&=& \frac{u}{\left(u^2 +1\right)^n} + 2n \left[ I_n \left(u\right) + I_{n+1} \left(u\right)\right]\nonumber
\end{eqnarray}

En cosecuencia
\begin{equation}
I_{n+1} \left(u \right) = \frac{2n-1}{2n}I_n \left(u\right) + \frac{1}{2n}\frac{u}{\left(u^2 +1 \right)^n}, \qquad \forall \: n \in \mathbb{N}
\end{equation}
F\'ormula que permite calcular recursivamente\footnote{\emph{``To iterate is human, to recurse divine.'' }L. Peter Deutsch } $I_n \left(u \right)$ para todo $n \in \mathbb{N}$ a partir de $I_1 \left(u \right) = \arctan u$ . Por lo tanto, hemos demostrado que  \emph{la integral de una funci\'on racional es una funci\'on elemental expresable en t\'erminos de funciones racionales, logaritmos y arcotangentes}.

%%%

\subsubsection{Factores lineales no repetidos}
Si $Q_m \left(x\right)$ tiene $m$ ceros reales y simples, o sea, si su factorizaci\'on es de la forma
$$Q_m \left(x\right)= \left(x-x_1\right)\left(x-x_2\right)\ldots\left(x-x_{m-1}\right)\left(x-x_m\right)$$
entonsces $\frac{P_n\left(x\right)}{Q_m \left(x\right)}$ se puede descomponer en las fracciones \emph{elementales simples}
$$\frac{P_n\left(x\right)}{Q_m \left(x\right)} = \frac{A_1}{\left(x-x_1\right)}+\frac{A_2}{\left(x-x_2\right)}+ \ldots + \frac{A_{m-1}}{\left(x-x_{m-1}\right)}+\frac{A_m}{\left(x-x_m\right)}$$
Donde $A_1, \ldots, A_m$ se calculan por la f\'ormula
\begin{equation}
A_k = \lim_{x\rightarrow x_k} \frac{P_n \left(x\right)\left(x-x_k\right)}{Q_m\left(x\right)} \:, \qquad k=1,2, \ldots , m
\end{equation}
\begin{ejm}
Calcular $\int \frac{x-1}{x^3-x^2-2x}dx$. Primero encontramos las fracciones mas elementales
$$\frac{x-1}{x^3-x^2-2x}=\frac{x-1}{x\left(x-2\right)\left(x+1\right)}=\frac{A_1}{x}+\frac{A_2}{x-2}+\frac{A_3}{x+1}$$
Luego, aplicando lo mencionado anteriormente:
$$A_1=\lim_{x\rightarrow 0} \frac{\left(x-1\right)x}{x\left(x-2\right)\left(x+1\right)}=\frac{1}{2} \: , \qquad A_2= \lim_{x\rightarrow 2} \frac{\left(x-1\right)\left(x-2\right)}{x\left(x-2\right)\left(x+1\right)} = \frac{1}{6}$$
$$A_3 = \lim_{x\rightarrow -1} \frac{\left(x-1\right)\left(x+1\right)}{x\left(x-2\right)\left(x+1\right)}= -\frac{2}{3} \: .$$
Finalmente, integrando:
$$\int \frac{x-1}{x^3-x^2-2x}dx = \int \left(\frac{1/2}{x}+\frac{1/6}{x-2} +\frac{-2/3}{x+1}\right)dx  =\frac{1}{2}\ln{|x|} +\frac{1}{6}\ln{|x-2|} -\frac{2}{3}\ln{|x+1|}$$
Otra manera\footnote{F\'acilmente demostrable teniendo en cuenta que $Q'_n\left(x\right)=\sum^{n}_{i=1}\prod_{j\neq 1}\left(x-x_j\right)$} de calcular los coeficientes ser\'ia de la siguiente forma:
$$A_k=\frac{P_n\left(x_k\right)}{Q'_m\left(x_k\right)}$$

Ahora para el ejemplo anterior tenemos:
$$A_k= \left[\frac{x-1}{\left(x^3-x^2-2x\right)'}\right]_{x=k}= \left[\frac{x-1}{3x^2-2x-2}\right]_{x=k}$$
luego
$$A_1=\left[\frac{x-1}{3x^2-2x-2}\right]_{x=0}=\frac{1}{2} \: , \qquad A_2=\left[\frac{x-1}{3x^2-2x-2}\right]_{x=2}=\frac{1}{6}$$
$$A_3=\left[\frac{x-1}{3x^2-2x-2}\right]_{x=-1}=-\frac{2}{3}$$
\end{ejm}

%%%

\subsubsection{Factores lineales repetidos}
Si $Q_m \left(x\right)$ se puede descomponear en factores lineales y algunos se repiten. Supongamos que $\left(x-x_1\right)$ se repite $k$ veces, o sea se descompone en: 
$$Q_m \left(x\right)=\left(x-x_1\right)^k H_{m-k}\left(x\right)$$
con $H_{m-k} \left(x_1\right) \neq 0$, entonces $\frac{P_n \left(x\right)}{Q_m \left(x\right)}$ se descompone en la forma:
$$\frac{P_n \left(x\right)}{Q_m \left(x\right)} = \frac{P_n \left(x\right)}{\left(x-x_1\right)^k H_{m-k}\left(x\right)} = \frac{A_1}{\left(x-x_1\right)^k} + \frac{A_2}{\left(x-x_1\right)^{k-1}} + \ldots + \frac{A_k}{\left(x-x_1\right)}+ \frac{R\left(x\right)}{H_{m-k}\left(x\right)}$$
Multiplicando por $\left(x-x_1\right)$, obtenemos:
$$\frac{P_n \left(x\right)}{Q_m \left(x\right)} = A_1+A_1\left(x-x_1\right)+\ldots+A_k\left(x-x_k\right)^{k-1}+ \left(x-x_1\right)^k \frac{R\left(x\right)}{H_{m-k} \left(x\right)}$$
De aqui se deduce que $A_1=\frac{P_n\left(x_1\right)}{H_{m-k}\left(x_1\right)}$, $A_2=\frac{d}{dx}\left(\frac{P_n\left(x_1\right)}{H_{m-k}\left(x_1\right)}\right)$; en general:
\begin{equation}
A_k = \frac{1}{\left(k-1\right)!}\frac{d^{k-1}}{dx^{k-1}}\left(\frac{P_n\left(x_1\right)}{H_{m-k}\left(x_1\right)}\right)
\end{equation}
\begin{ejm}
Calcular $\int \frac{5x^2-23x}{\left(2x-2\right)\left(2x+4\right)^4} \: dx$, esta es una integral un poco m\'as complicada, primero hallamos las fracciones parciales:
\begin{eqnarray}
\lefteqn {\frac{5x^2-23x}{\left(2x-2\right)\left(2x+4\right)^4} = \frac{1}{32}\frac{5x^2-23x}{\left(x-1\right)\left(x+2\right)} }  \nonumber \\
&& \qquad \qquad \qquad \: \: = \frac{1}{32}\left[\frac{A_1}{\left(x+2\right)^4} + \frac{A_2}{\left(x+2\right)^3} + \frac{A_3}{\left(x+2\right)^2} + \frac{A_4}{\left(x+2\right)} +\frac{B_1}{\left(x-1\right)}\right] \nonumber
\end{eqnarray}
Hallamos los coeficientes:
\begin{eqnarray}
	&& A_1=\frac{1}{0!}\left[\frac{5x^2-23x}{x-1}\right]_{x=-2} = -22 \nonumber \\
	&& A_2=\frac{1}{1!}\left[\frac{d}{dx}\left(\frac{5x^2-23x}{x-1}\right)\right]_{x=-2} = \left[\frac{5x^2-10x+23}{\left(x-1\right)^2}\right]_{x=-2} = 7\nonumber \\
	&& A_3=\frac{1}{2!}\left[\frac{d^2}{dx^2}\left(\frac{5x^2-23x}{x-1}\right)\right]_{x=-2} = \frac{1}{2!}\left[\frac{-36}{\left(x-1\right)^3}\right]_{x=-2} = \frac{2}{3} \nonumber \\
	&& A_4=\frac{1}{3!}\left[\frac{d^3}{dx^3}\left(\frac{5x^2-23x}{x-1}\right)\right]_{x=-2} = \frac{1}{3!}\left[\frac{108}{\left(x-1\right)^4}\right]_{x=-2} = \frac{2}{9} \nonumber \\
	&& B_1=\frac{1}{0!}\left[\frac{5x^2-23x}{\left(x+2\right)^4}\right]_{x=1} = -\frac{2}{9} \nonumber
\end{eqnarray}
Finalmente integramos:
{\footnotesize
\begin{eqnarray}
	\int \frac{5x^2-23x}{\left(2x-2\right)\left(2x+4\right)^4}dx &=& \frac{1}{32}\int \left[\frac{-22}{\left(x+2\right)^4} + \frac{7}{\left(x+2\right)^3} + \frac{2/3}{\left(x+2\right)^2} + \frac{2/9}{\left(x+2\right)} +\frac{-2/9}{\left(x-1\right)}\right] dx \nonumber \\
	&=& \frac{1}{32}\left[\frac{22}{3\left(x+2\right)^3} - \frac{7}{2\left(x+2\right)^2} - \frac{2}{3\left(x+2\right)} + \frac{2}{9}\ln{|x+2|} - \frac{2}{9}\ln{|x-1|}\right] \nonumber
\end{eqnarray}
}
\end{ejm}
%\subsubsection{Factores cuadr\'aticos}

%\subsubsection{Factores cuadr\'aticos repetidos}

\subsection{Integrales reducibles a integrales de funciones racionales}
\subsubsection{De la forma $\int R\left(e^x\right)dx$}
Las integrales de la forma $\int R\left(e^x\right)dx $ donde $R$ es una funci\'on racional se reducen a la integral de una funci\'on racional mediante el cambio de variable $u=e^x$:
\begin{equation}
	\int R\left(e^x\right)dx = \left\{ \begin{array}{c}
	u=e^x \\ du=e^xdx
	\end{array} \right\} =\left[\int \frac{R\left(u\right)}{u}dx\right]_{u=e^x} 
\end{equation}
\subsubsection{De la forma $\int R\left(\sen{x},\cos{x}\right)dx$}\label{itrig}
La integral de la forma:$\int R\left(\sen{x},\cos{x}\right)dx$ donde $R\left(s,t\right)$ es una funci\'on racional de dos variables $s$ y $t$, tambi\'en se reduce a la integral de una funci\'on racional, mediante el cambio de variable $u=\tan{x}$:
\begin{eqnarray}
	\int R\left(\sen{x},\cos{x}\right)dx &=& \left\{ \begin{array}{lr}
		u=\tan{x} & dx=\frac{2du}{1+u^2} \\
	\sen{x}=\frac{2u}{1+u^2} & \cos{x}=\frac{1-u^2}{1+u^2}\end{array} \right\}  \nonumber \\ 
	&=& \int\frac{2}{1+u^2}R\left(\frac{2u}{1+u^2},\frac{1-u^2}{1+u^2}\right)dx
\end{eqnarray}
Existen varios tipos de integrales trigonom\'etricas que se pueden \emph{racionalizar} con cambios m\'as sencillos. Ellas son las siguientes:
\begin{eqnarray}
\int f(\sen x ,\cos x) dx \textrm{, donde } f(-\sen x , \cos x)= -f(\sen x ,\cos x)\textrm{, cambio } t=\cos x \nonumber\\
\int f(\sen x ,\cos x) dx \textrm{, donde } f(\sen x , -\cos x)= -f(\sen x ,\cos x)\textrm{, cambio } t=\sen x \nonumber\\
\int f(\sen x ,\cos x) dx \textrm{, donde } f(-\sen x , -\cos x)= f(\sen x ,\cos x)\textrm{, cambio } t=\tan x \nonumber
\end{eqnarray}
\subsubsection{De la forma $\int R\left(x,\sqrt{\frac{ax+b}{cx+d}}\right)dx$}
Las integrales de la forma
$$\int R\left(x,\sqrt{\frac{ax+b}{cx+d}}\right)dx$$
donde de nuevo $R$ es una funci\'on racional de dos variable y
$$\Delta =ad-bc \neq 0$$
Tambien se reducen a la integral de una funci\'on racional mediante el cambio de variable
$$u= \sqrt{\frac{ax+b}{cx+d}} \Longrightarrow x=\frac{du^2-b}{a-cu^2}$$
en efecto
$$du=\frac{\Delta}{2u}\frac{dx}{(cx+d)^2} = \frac{(a-cu^2)^2}{2\Delta u}dx$$
de donde
$$\int R\left(x,\sqrt{\frac{ax+b}{cx+d}} \right)dx = 2\Delta\left[\int \frac{u}{(a-cu^2)^2}R \left(\frac{du^2-b}{a-cu^2},u\right)du\right]_{u=\sqrt{\frac{ax+b}{cx+d}}}$$
En particular, si R es una funci\'on racional de dos variables, la integral
$$\int R(x,\sqrt{ax+b})dx \qquad a \neq 0$$
se convierte mediante el cambio
$$u=\sqrt{ax+b}$$
en lka integral de una funci\'on racional:
$$\int R(x,\sqrt{ax+b})dx= \frac{2}{a}\left[\int R\left(\frac{u^2 -b}{a},u\right)udu\right]_{u=\sqrt{ax+b}}$$
\subsubsection{De la forma $\int R\left(x,\sqrt{ax^2+bx+c}\right)dx$}
Las integrales del tipo
$$\int R\left(x,\sqrt{ax^2+bx+c}\right)dx \qquad a\neq 0, \: \Delta=b^2-4ac\neq 0 $$
Donde $R$ es una funci\'on racional de dos variables, no se transformanm en integrales de funciones racionales mediante el cambio $u=\sqrt{ax^2+bx+c}$,
Para racionalizar esta integral, se efectua en primer lugar un cambio de variable lieal
$$ax+\frac{b}{2}=\frac{\sqrt{\Delta}}{2}u$$
que lo transforman en los siguientes tipos mas sencillos:
\begin{align}
& \int \tilde{R}\left(u, \sqrt{1-u^2}\right)du &&(a<0,\: \Delta>0) \label{eq:tip1}\\
& \int \tilde{R}\left(u, \sqrt{u^2-1}\right)du &&(a>0,\: \Delta>0) \label{eq:tip2}\\
& \int \tilde{R}\left(u, \sqrt{1+u^2}\right)du &&(a>0,\: \Delta<0) \label{eq:tip3}
\end{align}
Donde $\tilde{R}$ es de nuevo una funci\'on racional de dos variables. 

Si la integral es del tipo \eqref{eq:tip1}, el cambio de variable
$$u=\sen t,\qquad t\in\left[\frac{\pi}{2},-\frac{\pi}{2}\right]$$
(\'o $u=\cos t$) la reduce a una integral trigonom\'etrica (secci\'on \ref{itrig}):
$$\int \tilde{R}\left(u,\sqrt{1-u^2}\right)du=\left[\int \tilde{R}\left(\sen x, \cos x\right)\cos t dt\right]_{t=\arcsen x}$$
donde se ha tenido en cuenta que $\cos t \geq 0$ si $-\frac{\pi}{2}\leq t\leq\frac{\pi}{2}$

An\'alogamente si la integral es del tipo \eqref{eq:tip2} el cambio de variable
$$u=\sec t, \qquad t\in \left[0,\pi\right]-\frac{\pi}{2}$$
la reduce a una integral trigonom\'etrica (secci\'on \ref{itrig}):
$$\int \tilde{R}\left(u,\sqrt{u^2 -1}\right)=\left[\int \tilde{R}\left(\sec t, \pm \tan t\right)\sec t \tan t dt \right]_{t=\arcsec u}$$
donde el signo $+$ (resp. $-$) corresponde a $u \geq 1$ (resp. $u \leq -1$).

Finalmente una integral del tipo \eqref{eq:tip3} se convierte de nuevo en una integral trigonom\'etrica (secci\'on \ref{itrig}) mediante el cambio
$$u = \tan t, \qquad t\in \langle -\frac{\pi}{2},\frac{\pi}{2}\rangle$$
En efecto
$$\int \tilde{R}\left(u,\sqrt{1+u^2}\right)du=\left[\int \tilde{R}\left(\tan t, \sec t\right)\sec^2 t dt\right]_{t=\arctan u}$$ 
%%%%%%%%%%%%%%%%%%%%%%%%%%%%%%%%%%%%%%%%%%%%%%%%%%%%%%%%%%%%%%%%%%%%%%%%%%%%%%%%%
%%%%%%%%%%%%%%%%%%%%%%%%%%%%%%%%%%%%%%%%%%%%%%%%%%%%%%%%%%%%%%%%%%%%%%%%%%%%%%%%%
\section{Anexo}
\subsection{Aplicaciones de la Integral Definida}
\begin{enumerate}

\item \textbf{\'Area de una figura plana asociada a una curva\footnote{Usando el teorema de Green, el \'area definida por el contorno $C$ simple, cerrado, positivamente-orientado es: $$A=\oint_C x dy=\oint_C y dx$$}}
\begin{align}
	&A=\int^b_a |f(x)|dx && \text{Expl\'icita}\\
	&A=\left|\int^{t_1}_{t_0} y(t)x'(t)\right|dt && \text{Param\'etrica}\\
	&A=\frac{1}{2}\int^{\theta_1}_{\theta_0} \rho (\theta) ^2 d\theta && \text{Polar}
\end{align}
%%
\item \textbf{Longitud de arco de una curva plana}
\begin{align}
	&L=\int^b_a \sqrt{1+f'(x)^2}dx && \text{Expl\'icita}\\
	&L=\int^{t_1}_{t_0} \sqrt{x'(t)^2+y'(t)^2}dt && \text{Param\'etrica}\\
	&L=\int^{\theta_1}_{\theta_0} \sqrt{\rho (\theta) ^2+\rho' (\theta) ^2} d\theta && \text{Polar}
\end{align}
%%
\item \textbf{Volumen del s\'olido de revoluci\'on asociado a una curva plana}

\begin{align}
	&V=\int^b_a f(x)^2 dx &&\text{Alrededor del eje }X\\
	&V= 2\pi\int^b_a x|f(x)|dx &&\text{Alrededor del eje }Y\\
	&V=\pi\left|\int^{t_1}_{t_0} y(t)^2 x'(t) dt\right| && \text{Param\'etrica}\\
	&V=\frac{2\pi}{3}\int^{\theta_1}_{\theta_0}\rho(\theta)^3 \sen \theta d\theta  && \text{Polar}
\end{align}
%%
\item \textbf{Volumen de un solido de \'area conocida}
\begin{equation}
	V=\int^b_a S(x)dx
\end{equation}
%%
\item \textbf{\'Area de una superficie de revoluci\'on alrededor del eje $x$ generada por una curva plana}
\begin{align}
	&A=2\pi\int^b_a |f(x)|\sqrt{1+f'(x)^2}dx &&\text{Expl\'icita}\\
	&A=2\pi \int^{t_1}_{t_0}|y(t)| \sqrt{x'(t)^2+y'(t)^2}dt &&\text{Param\'etrica}\\
	&A=2\pi\int^{\theta_1}_{\theta_0} \rho (\theta) |\sen{\theta}| \sqrt{\rho (\theta)^2+\rho'(\theta) ^2} d\theta &&\text{Polar}
\end{align}		
%%
\item \textbf{Centro de gravedad (baricentro) de una curva plana con una densidad de masa $\delta$}

		\begin{enumerate}
			\item[] Explicita $(x_0 ,x_0)$ 
\begin{eqnarray}
x_0=\frac{\int^b_a x\delta(x)\sqrt{1 +f'(x)^2} dx}{\int^b_a \delta(x)\sqrt{1 +f'(x)^2} dx}\\
y_0=\frac{\int^b_a f(x)\delta(x)\sqrt{1 +f'(x)^2} dx}{\int^b_a \delta(x)\sqrt{1 +f'(x)^2} dx}
\end{eqnarray}

			\item[] Param\'etrica $(x_0,y_0)$
\begin{eqnarray}
x_0=\frac{\int^{t_1}_{t_0} x(t)\delta(t) \sqrt{x'(t)^2+y'(t)^2}dt}{\int^{t_1}_{t_0} \delta(t) \sqrt{x'(t)^2+y'(t)^2}dt}\\
y_0=\frac{\int^{t_1}_{t_0} y(t)\delta(t) \sqrt{x'(t)^2+y'(t)^2}dt}{\int^{t_1}_{t_0} \delta(t) \sqrt{x'(t)^2+y'(t)^2}dt}
\end{eqnarray}
			\item[] Polar $(x_0,y_0)$ 
\begin{eqnarray}
x_0=\frac{\int^{\theta_1}_{\theta_0}\delta(\theta)\rho(\theta)\cos\theta\sqrt{\rho(\theta)^2+\rho'(\theta)^2}d\theta}{\int^{\theta_1}_{\theta_0} \delta(\theta)\sqrt{\rho(\theta) ^2+\rho'(\theta)^2}d\theta}\\
y_0=\frac{\int^{\theta_1}_{\theta_0} \delta(\theta)\rho (\theta)\sen \theta \sqrt{\rho(\theta)^2+\rho'(\theta)^2}d\theta}{\int^{\theta_1}_{\theta_0}\delta(\theta)\sqrt{\rho(\theta)^2+\rho'(\theta)^2}d\theta}
\end{eqnarray}
		\end{enumerate}

	\item \textbf{Momento de Inercia de una curva plana con densidad $\delta$, con respecto a una recta o un punto}
		\begin{enumerate}
			\item[] Expl\'icita: $d(x)$ es la distancia del punto $(x,f(x))$ a la recta o al punto.
\begin{equation}
I=\int^b_a d(x)^2\delta(x)\sqrt{1+f'(x)^2}dx
\end{equation}
			\item[] Param\'etrica: $d(t)$ es la distancia del punto $(x(t),y(t))$ a la recta o al punto. 
\begin{equation}
I=\int^{t_1}_{t_0} d(t)^2 \delta(t) \sqrt{x'(t)^2+y'(t)^2}dt
\end{equation}
			\item[] Polar: $d(\theta)$ es la distancia del punto $(\theta,\rho(\theta))$ a la recta o al punto. 
\begin{equation}
I=\int^{\theta_1}_{\theta_0} d(\theta)^2\delta(\theta)\sqrt{\rho(\theta)^2+\rho'(\theta)^2}d\theta
\end{equation}
		\end{enumerate}

	\end{enumerate}
\subsection{Tabla de integrales}
Coming soon :-), mientras tanto, puedes echarle un vistazo al libro \emph{CRC Standard Mathematical Tables and Formulae}\footnote{I love this book} con una tabla de nada menos 708 integrales, eso s\'i bien ordenaditas ;-) 

\newpage

%%%%%%%%%%%%%%%%%%%%%%%%%%%%%%%%%%%%%%%%%%%%%%%%%%%%%%%%%%%%%%%%%%%%%%%%%
%%%%%%%%%%%%%%%%%%%%%%%%%%%%%%%%%%%%%%%%%%%%%%%%%%%%%%%%%%%%%%%%%%%%%%%%%


\end{document}
